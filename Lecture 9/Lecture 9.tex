\documentclass[fleqn]{article}
\usepackage{amsmath, amssymb, amsthm}
%\usepackage{gensymb}
\usepackage{commath}
\usepackage{siunitx}
\usepackage{tikz, pgfplots}
\usetikzlibrary{calc, hobby}
\usepackage{graphicx}
\usepackage{hyperref}
\usepackage{datetime}
\usepackage{ulem}
\usepackage{xcolor}
\usepackage{enumerate}
\setcounter{secnumdepth}{4}

\newcommand\numberthis{\addtocounter{equation}{1}\tag{\theequation}}

\newcommand{\AxisRotator}[1][rotate=0]{%
	\tikz [x=0.25cm,y=0.60cm,line width=.2ex,-stealth,#1] \draw (0,0) arc (-150:150:1 and 1);%
}

\theoremstyle{definition}
\newtheorem{example}{Example}
\newtheorem{definition}{Definition}

\theoremstyle{theorem}
\newtheorem{theorem}{Theorem}

\newenvironment{solution}
{\begin{proof}[Solution]\let\qed\relax}
	{\end{proof}}

\newcommand{\curl}{\mathrm{curl\,}}

%\renewcommand{\int}{\int\displaylimits}

%opening
\title{Lecture 9}
\author{Aakash Jog}
\date{\formatdate{25}{11}{2014}}

\begin{document}

\maketitle
%\setlength{\mathindent}{0pt}

\tableofcontents

\newpage
\section{Curl}

\begin{tikzpicture}
	\draw [<->] (-1,0) -- (5,0);
	\draw [<->] (0,-1) -- (0,5);
	
	\fill (1,0) circle [radius=1pt] node [below] {$x_0$};
	\fill (4,0) circle [radius=1pt] node [below] {$x_0 + \dif x$};
	\fill (0,1) circle [radius=1pt] node [left] {$y_0$};
	\fill (0,4) circle [radius=1pt] node [left] {$y_0 + \dif y$};
	
	\draw [->] (1,1) -- (4,1);
	\draw [->] (4,1) -- (4,4);
	\draw [->] (4,4) -- (1,4);
	\draw [->] (1,4) -- (1,1);
\end{tikzpicture}

\begin{align*}
		\oint \overrightarrow{F} \cdot \dif \overrightarrow{r} 
		&= \int \overrightarrow{F}_x (x_0 + \dfrac{\dif x}{2}, y_0, z_0) \dif x 
		+ \int \overrightarrow{F}_y (x_0 + \dif x, y_0 + \dfrac{\dif y}{2}, z_0) \dif y\\
		&- \int \overrightarrow{F}_x (x_0 + \dfrac{\dif x}{2}, y_0 + \dif y, z_0) \dif x 
		- \int \overrightarrow{F}_y (x_0, y_0 + \dfrac{\dif y}{2}, z_0) \dif y \\
		&= \Big( 
		\int \overrightarrow{F}_x (x_0 + \dfrac{\dif x}{2}, y_0, z_0)
		- \int \overrightarrow{F}_x (x_0 + \dfrac{\dif x}{2}, y_0 + \dif y, z_0) 
		\Big) \dif x \\
		&+ \Big(
		\int \overrightarrow{F}_y (x_0 + \dif x, y_0 + \dfrac{\dif y}{2}, z_0)
		- \int \overrightarrow{F}_y (x_0, y_0 + \dfrac{\dif y}{2}, z_0) 
		\Big) \dif y \\
		&= \dfrac{\int \overrightarrow{F}_x (x_0 + \dfrac{\dif x}{2}, y_0, z_0)
		- \int \overrightarrow{F}_x (x_0 + \dfrac{\dif x}{2}, y_0 + \dif y, z_0)}{\dif y} \dif x \dif y \\
		&+ \dfrac{\int \overrightarrow{F}_y (x_0 + \dif x, y_0 + \dfrac{\dif y}{2}, z_0)
		- \int \overrightarrow{F}_y (x_0, y_0 + \dfrac{\dif y}{2}, z_0)}{\dif x} \dif x \dif y \\
		&= 
		- \dpd{F(x)}{y} \dif x \dif y
		+ \dpd{F_y}{x} \dif x \dif y \\
		&= \left(\dpd{F_y}{x} - \dpd{F_x}{y}\right) \dif x \dif y
		\intertext{As $\overrightarrow{F}$ is conservative,}
		\oint \overrightarrow{F} \cdot \dif \overrightarrow{r} &= 0 \\
		\therefore \dpd{F_y}{x} - \dpd{F_x}{y} &= 0 \\
		\therefore \dpd{F_z}{x} - \dpd{F_x}{z} &= 0 \\
		\therefore \dpd{F_z}{y} - \dpd{F_y}{z} &= 0 \\
\end{align*}

\begin{definition}
	\begin{align*}
		\curl \overrightarrow{F} &\doteq \overrightarrow{\nabla} \times \overrightarrow{F} \\
		&=
		\begin{vmatrix} 
			\hat{x} & \hat{y} & \hat{z} \\
			\dpd{}{x} & \dpd{}{y} & \dpd{}{z} \\
			F_x & F_y & F_z \\
		\end{vmatrix}
	\end{align*}
	If $\overrightarrow{F}$ is conservative, $\curl \overrightarrow{F} = 0$.
\end{definition}

\section{Equilibria}

\begin{equation*}
	F = -\dod{U}{x}
\end{equation*}
Therefore, the force is pointing to the direction in which $U$ decreases.

\begin{tikzpicture}
	\draw [<->] (-1,0) -- (7,0) node [right] {$x$};
	\draw [<->] (0,-1) -- (0,3) node [above] {$U$};
	
	\coordinate (a) at (1,1);
	\coordinate (b) at (2,4);
	\coordinate (c) at (3,2);
	\coordinate (d) at (4,3);
	\coordinate (e) at (5,1);
	
	\draw plot [smooth, tension=2] coordinates {(a) (b) (c) (d) (e)};
\end{tikzpicture}\\
At the local extrema of $U$, the force is always zero. \\
In the neighbourhood of the point with the local minima, all forces are pointing towards the point, and at the local maxima, they are pointing away from it.\\
Therefore at the point where $U$ is minimum, there is stable equilibrium. At the point where $U$ is maximum, there is unstable equilibrium.

\begin{example}
	\begin{tikzpicture}
		\draw [->, very thin, lightgray] (0,0) -- (5,0) node [right] {$x$};
		\draw [->, very thin, lightgray] (0,0) -- (0,-5) node [below] {$y$};
		
		\draw [dashed, red] (-6,0) -- (0,0) node [above] {$U = 0$};
		
		\coordinate (a) at (-4,{-0.25*(4^2)});
		\coordinate (b) at (4,{-0.25*(4^2)});
		
		\draw plot [domain=-6:6] (\x, {-0.25*pow(\x,2)});
		
		\fill (a) circle [radius = 4pt];
		\fill (b) circle [radius = 4pt];
		
		\draw [decorate, decoration={coil, aspect=1}] (a) -- (b);
	\end{tikzpicture}
	\begin{align*}
		y &= b x^2
	\end{align*}
	Find $x_{\text{eq}}$.
\end{example}

\begin{solution}[Solution using energy]
	\begin{align*}
		U &= - 2 m g y + \dfrac{1}{2} k (2x - l)^2 \\
		&= - 2 m g bx^2 + \dfrac{1}{2} k (2x - l)^2
		\intertext{Differentiating w.r.t. $x$,}
		0 = U' &= - 4 m g b x_{\text{eq}} + 2 k (2x - ) \\
		\therefore x_{\text{eq}} = \dfrac{k l}{2(k - m g b)}
	\end{align*}
\end{solution}

%\begin{solution}[Solution using forces]
%	\begin{tikzpicture}
%		\draw [->, very thin, lightgray] (0,0) -- (5,0) node [right] {$x$};
%		\draw [->, very thin, lightgray] (0,0) -- (0,-5) node [below] {$y$};
%		
%		\draw [dashed, red] (-6,0) -- (0,0) node [above] {$U = 0$};
%		
%		\coordinate (a) at (-4,{-0.25*(4^2)});
%		\coordinate (b) at (4,{-0.25*(4^2)});
%		
%		\draw plot [domain=-6:6] (\x, {-0.25*pow(\x,2)});
%		
%		\fill (a) circle [radius = 4pt];
%		\fill (b) circle [radius = 4pt];
%		
%		\draw [decorate, decoration={coil, aspect=1}] (a) -- (b);
%	\end{tikzpicture}
%	
%\end{solution}

\begin{example}
	\begin{tikzpicture}
		\draw [->, very thin, lightgray] (0,0) -- (5,0) node [right] {$x$};
		\draw [->, very thin, lightgray] (0,0) -- (0,-5) node [below] {$y$};
	
		\draw (0,0) -- ++(210:8) -- ++(0:{8*cos(30)}) -- cycle;	
		
		\draw [dashed, very thin, red] (-4,0) -- (4,0) node [above] {$U = 0$};
		
		\draw [ultra thick] (210:3) -- (0,0) -- (-90:2);
		
		\draw [|<->|, xshift=10] (0,0) -- (-90:2) node [midway, fill=white] {$y$};
		\draw [|<->|, xshift=-12, yshift=12] (0,0) -- (210:3) node [midway, fill=white] {$l - y$}; %label along line??? 
	\end{tikzpicture}
\end{example}

\begin{solution}
	\begin{align*}
		U_{\text{right side}} &= \int \dif m g (-\tilde{y}) \\
		&= \int_{0}^{y} - \left(\dfrac{m_0}{l} \dif \tilde{y}\right) g \tilde{y}\\
		&= - \dfrac{m_0 g}{l} \int_{0}^{y} \tilde{y} \dif \tilde{y} \\
		&= - \dfrac{m_0 g}{l} \dfrac{y^2}{2} \\
		&= - \left(\dfrac{m_0}{l} y\right) g \left(\dfrac{y}{2}\right)
	\end{align*}
	\begin{align*}
		U_{\text{left side}} &= \int \dif m g (-\tilde{y} \sin \theta)\\
		&= \int_{0}^{l - y} - \left(\dfrac{m_0}{l} \dif \tilde{y}\right) g \tilde{y} \sin \theta\\
		&= - \dfrac{m_0 g}{l} \int_{0}^{l - y} \tilde{y} \dif \tilde{y} \sin \theta\\
		&= - \dfrac{m_0 g \sin \theta}{l} \dfrac{(l - y)^2}{2} \\
		&= - \left(\dfrac{m_0}{l} (l - y)\right) g \left(\dfrac{(l - y) \sin \theta}{2}\right)
	\end{align*}
	\begin{align*}
		\therefore U &= U_{\text{right side}} + U_{\text{left side}}\\
		&= - \dfrac{m_0 g}{2l} y^2 - \dfrac{m_0 g \sin \theta}{2l} (l - y)^2\\
		\intertext{Minimizing $U$,}
		y_{\text{eq}} &= \sin \theta (l - y)
	\end{align*}
\end{solution}

\section{Principle of Conservation of Momentum}

\begin{align*}
	m \ddot{\overrightarrow{r_1}} &= \overrightarrow{F_{1,\text{ext}}} + \overrightarrow{F_{1,2}}\\
	m \ddot{\overrightarrow{r_2}} &= \overrightarrow{F_{2,\text{ext}}} + \overrightarrow{F_{2,1}}\\
	\therefore m \ddot{\overrightarrow{r_1}} + m \ddot{\overrightarrow{r_2}} &= \overrightarrow{F_{1,\text{ext}}} + \overrightarrow{F_{2,\text{ext}}}\\
	\therefore \sum m_i \ddot{\overrightarrow{r_i}} &= \sum \overrightarrow{F_{i,\text{ext}}}\\
	\intertext{If $\sum \overrightarrow{F_{\text{ext}}} = 0$,}
	\sum m_i \ddot{\overrightarrow{r_i}} &= 0\\
	\therefore \sum m_i \dod{\overrightarrow{v_i}}{t} &= 0\\
	\therefore \dod{}{t}\left(\sum m_i \overrightarrow{v_i}\right)&= 0\\
	\therefore \overrightarrow{p_{\text{total}}} = \sum m_i \overrightarrow{v_i} &= \text{ constant }
\end{align*}

\section{Centre of Mass}

\begin{align*}
	\sum m_i a_i &= \dod{}{t} \overrightarrow{p_{\text{total}}} = \sum \overrightarrow{F_{\text{ext}}} \\
	\therefore \left(\sum m_i\right) a_{\text{total}} &= \sum m_i \ddot{\overrightarrow{r_i}} \\
	\therefore a_{\text{total}} &= \dfrac{\sum m_i \ddot{\overrightarrow{r_i}}}{\sum m_i} \\
	&= \dod[2]{}{t} \left(\dfrac{\sum m_i \overrightarrow{r_i}}{\sum m_i} \right)\\
	&\doteq \ddot{\overrightarrow{r_{\text{COM}}}}\\
	\overrightarrow{r_{\text{COM}}} &\doteq \dfrac{\sum m_i \overrightarrow{r_i}}{\sum m_i}
\end{align*}

\end{document}