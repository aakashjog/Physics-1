\documentclass[fleqn, a4paper, 12pt]{article}
\usepackage{amsmath, amssymb, amsthm}
\usepackage{gensymb}
\usepackage{commath}
\usepackage{xcolor}
\usepackage{cancel}
\usepackage{siunitx}
\usepackage{tikz, pgfplots}
	\usetikzlibrary{calc, hobby, patterns, intersections}
\usepackage{graphicx}
\usepackage{hyperref}
\usepackage{datetime}
\usepackage{ulem}
\usepackage{xfrac}
\usepackage{asymptote}
\usepackage{enumerate}
\setcounter{secnumdepth}{4}
\newcommand\numberthis{\addtocounter{equation}{1}\tag{\theequation}}

\newcommand{\AxisRotator}[1][rotate=0]{%
	\tikz [x=0.25cm,y=0.60cm,line width=.2ex,-stealth,#1] \draw (0,0) arc (-150:150:1 and 1);%
}

\theoremstyle{definition}
\newtheorem{example}{Example}
\newtheorem{definition}{Definition}

\theoremstyle{theorem}
\newtheorem{theorem}{Theorem}

\newenvironment{solution}
{\begin{proof}[Solution]\let\qed\relax}
	{\end{proof}}

\newcommand{\curl}{\mathrm{curl\,}}

%\renewcommand{\int_{min}^{max}}{\int\displaylimits_{min}^{max}}

%opening
\title{Lecture 23}
\author{Aakash Jog}
\date{\formatdate{15}{1}{2015}}

\begin{document}

\maketitle
%\setlength{\mathindent}{0pt}

\tableofcontents

\newpage
\section{Gravitation}

\subsection{Kepler's Second Law}

\begin{tikzpicture}
	\def\a{5};
	\def\b{4};
	\def\angle{10};

	\coordinate (focus) at ({-sqrt(\a*\a - \b*\b)},0);

	\draw (0,0) circle [x radius = \a, y radius = \b];
	
	\draw [-stealth] (focus) -- ({\a*cos(\angle)},{\b*sin(\angle)}) node [midway, above] {$\overrightarrow{r}(t + \dif t)$};
	\draw [-stealth] (focus) -- (\a,0) node [midway, below] {$\overrightarrow{r}(t)$};
	\draw [-stealth] (\a,0) -- ({\a*cos(\angle)},{\b*sin(\angle)}) node [midway, right] {$\dif \overrightarrow{r}$};
\end{tikzpicture}

\begin{align*}
	\dif \overrightarrow{A} &= \dfrac{1}{2} \overrightarrow{r} \times \dif \overrightarrow{r}\\
	\therefore \dod{\overrightarrow{A}}{t} &= \dfrac{1}{2} \overrightarrow{r} \times \dod{\overrightarrow{r}}{t}\\
	&= \dfrac{1}{2} \overrightarrow{r} \times \overrightarrow{v}\\
	&= \dfrac{\overrightarrow{L}}{2m}
\end{align*}

\subsection{Newton's Law of Gravitation}

\begin{align*}
	F &= G \dfrac{m_1 m_2}{r^2}
\end{align*}

\begin{align*}
	\overrightarrow{F} &= \dfrac{c}{r^2} \hat{r}\\
	&= \dfrac{c}{x^2 + y^2 + z^2} \left( \dfrac{x}{\sqrt{x^2 + y^2 + z^2}}, \dfrac{y}{\sqrt{x^2 + y^2 + z^2}}, \dfrac{z}{\sqrt{x^2 + y^2 + z^2}} \right)\\
	&= \dfrac{c}{(x^2 + y^2 + z^2)^{\sfrac{3}{2}}} (x,y,z)\\
	\therefore \curl \left( \overrightarrow{F} \right) &= 0
\end{align*}
Therefore, $\overrightarrow{F}$ is conservative.

\begin{align*}
	\overrightarrow{F} &= - \overrightarrow{\nabla} U\\
	\dpd{U}{x} &= - \dfrac{c x}{(x^2 + y^2 + z^2)^{\sfrac{3}{2}}}\\
	\dpd{U}{y} &= - \dfrac{c y}{(x^2 + y^2 + z^2)^{\sfrac{3}{2}}}\\
	\dpd{U}{z} &= - \dfrac{c z}{(x^2 + y^2 + z^2)^{\sfrac{3}{2}}}
\end{align*}
Therefore,
\begin{align*}
	U &= \dfrac{c}{r} + d
\end{align*}
Considering the potential to be zero at $r = \infty$,
\begin{align*}
	U &= \dfrac{c}{r}\\
	&= -G\dfrac{m_1 m_2}{r}
\end{align*}

\end{document}