\documentclass[fleqn]{article}
\usepackage{amsmath, amssymb, amsthm}
%\usepackage{gensymb}
\usepackage{commath}
\usepackage{siunitx}
\usepackage{tikz, pgfplots}
\usetikzlibrary{calc, hobby}
\usepackage{graphicx}
\usepackage{hyperref}
\usepackage{datetime}
\usepackage{ulem}
\usepackage{xcolor}
\usepackage{enumerate}
\setcounter{secnumdepth}{4}

\newcommand\numberthis{\addtocounter{equation}{1}\tag{\theequation}}

\newcommand{\AxisRotator}[1][rotate=0]{%
	\tikz [x=0.25cm,y=0.60cm,line width=.2ex,-stealth,#1] \draw (0,0) arc (-150:150:1 and 1);%
}

\theoremstyle{definition}
\newtheorem{example}{Example}
\newtheorem{definition}{Definition}

\theoremstyle{theorem}
\newtheorem{theorem}{Theorem}

\newenvironment{solution}
{\begin{proof}[Solution]\let\qed\relax}
	{\end{proof}}

%opening
\title{Lecture 8}
\author{Aakash Jog}
\date{\formatdate{20}{11}{2014}}

\begin{document}

\maketitle
%\setlength{\mathindent}{0pt}

\tableofcontents

\newpage
\section{Power}

Power is defined to be the rate of work done.

\begin{align*}
	P &\doteq \dod{W}{t} \\
	&= \dfrac{\overrightarrow{F} \cdot \dif \overrightarrow{r}}{\dif t} \\
	&= \overrightarrow{F} \cdot \dod{\overrightarrow{r}}{t} \\
	&= \overrightarrow{F} \cdot \overrightarrow{v}
\end{align*}

\section{Conservative and Non-Conservative Forces}

\begin{example}
	Find the work done when the body moves along the paths shown.
	\begin{tikzpicture}		
		\draw [very thin, lightgray, step=1] (0,0) grid (4,4);
		\draw [<->] (-1,0) -- (5,0) node [right] {$x$};
		\draw [<->] (0,-1) -- (0,5) node [above] {$y$};
				
		\draw [green, ->] (0,0) -- (1,0);
		\draw [red, ->] (1,0) -- (1,4);
		
		\draw [blue, ->] (0,0) -- (1,4);
		
		\draw [orange, ->, domain=0:1] plot (\x, 4*\x^2);
	\end{tikzpicture}
	\begin{equation*}
		\overrightarrow{F} = 3 x^2 y^2 \hat{x} + 2 x^3 y \hat{y}
	\end{equation*}
\end{example}

\begin{solution}
	\begin{align*}
		\overrightarrow{F} &= 3 x^2 y^2 \hat{x} + 2 x^3 y \hat{y}\\
		\dif \overrightarrow{r} &= (\dif x, \dif y) \\
		\therefore \overrightarrow{F} \cdot \dif \overrightarrow{r} &= 3 x^2 y^2 \dif x + 2 x^3 y \dif y
	\end{align*}
	\begin{align*}
		W_1 &= \int\displaylimits_{(0,0)}^{(1,4)} \overrightarrow{F} \cdot \dif \overrightarrow{r} \\
		&= {\color{green} \int\displaylimits_{0}^{1} 3 \cdot x^2 \cdot 0^2 \cdot \dif x} + {\color{red} \int\displaylimits_{0}^{4} 2 \cdot 1^3 \cdot y \cdot \dif y} \\
		&= {\color{green} 0} + {\color{red} 16} \\
		&= 16
	\end{align*}
	{
		\color{blue}
		\begin{align*}
			W_2 &= \int\displaylimits_{(0,0)}^{(1,4)}  \overrightarrow{F} \cdot \dif \overrightarrow{r} \\
			y &= 4x \\
			\therefore W_2 &= \int\displaylimits_{0}^{1} 3x^2(4x)^2 \dif x + 2 x^3 (4x) (4 \dif x) \\
			&= 80 \int\displaylimits_{0}^{1} x^4 \dif x \\
			&= 80 \dfrac{1^5}{5} \\
			&= 16
		\end{align*}
	}
	{
		\color{orange}
		\begin{align*}
			y &= 4x^2 \\
			\therefore \dod{y}{x} &= 8x \\
			\therefore \dif y &= 8 x \dif x \\
			W_2 &= \int\displaylimits_{(0,0)}^{(1,4)}  \overrightarrow{F} \cdot \dif \overrightarrow{r} \\
			&= \int\displaylimits_{0}^{1} 3 x^2 (4 x^2)^2 \dif x + 2 x^3 (4x^2) (8 x \dif x) \\
			&= \int\displaylimits_{0}^{1} 112 x^6 \dif x \\
			&= 112 \dfrac{1^7}{7} \\
			&= 16
		\end{align*}
	}
\end{solution}

\begin{example}
	Find the work done when the body moves along the paths shown.
	\begin{tikzpicture}		
		\draw [very thin, lightgray, step=1] (0,0) grid (4,4);
		\draw [<->] (-1,0) -- (5,0) node [right] {$x$};
		\draw [<->] (0,-1) -- (0,5) node [above] {$y$};
	
		\draw [green, ->] (0,0) -- (1,0);
		\draw [red, ->] (1,0) -- (1,4);
	
		\draw [blue, ->] (0,0) -- (1,4);
		\end{tikzpicture}
	\begin{equation*}
		\overrightarrow{F} = 3 x^2 \hat{x} + 2 x^3 y \hat{y}
	\end{equation*}
\end{example}

\begin{solution}
	\begin{align*}
		\overrightarrow{F} &= 3 x^2 \hat{x} + 2 x^3 y \hat{y}\\
		\dif \overrightarrow{r} &= (\dif x, \dif y) \\
		\therefore \overrightarrow{F} \cdot \dif \overrightarrow{r} &= 3 x^2 \dif x + 2 x^3 y \dif y
	\end{align*}
	\begin{align*}
		W_1 &= \int\displaylimits_{(0,0)}^{(1,4)} \overrightarrow{F} \cdot \dif \overrightarrow{r} \\
		&= {\color{green} \int\displaylimits_{0}^{1} 3 \cdot x^2 \cdot \dif x} + {\color{red} \int\displaylimits_{0}^{4} 2 \cdot 1^3 \cdot y \cdot \dif y} \\
		&= {\color{green} 1} + {\color{red} 16} \\
		&= 16
	\end{align*}
	{
		\color{blue}
		\begin{align*}
			W_2 &= \int\displaylimits_{(0,0)}^{(1,4)}  \overrightarrow{F} \cdot \dif \overrightarrow{r} \\
			y &= 4x \\
			\therefore W_2 &= \int\displaylimits_{0}^{1} 3x^2 \dif x + 2 x^3 (4x) (4 \dif x) \\
			&= \int\displaylimits_{0}^{1} (3x^2 + 32x^4) \dif x \\
			&= \dfrac{37}{5} \\
		\end{align*}
	}
	It is evident that in this case, the work done is dependant on the path.
\end{solution}

\subsection{Potential Energy Corresponding to a 1D Force}

\begin{tikzpicture}
	\draw [<->] (-1,0) -- (5,0);
	\draw [<->] (0,-1) -- (0,1);
	
	\fill (2,0) circle [radius=2pt];
	\draw [|->|, yshift= -10] (0,0) -- (2,0) node [midway, below] {$x_A$};
	\fill (3,0) circle [radius=2pt];
	\draw [|->|, yshift= -20] (0,0) -- (3,0) node [midway, below] {$x_B$};
\end{tikzpicture}

\begin{align*}
	W &= \int\displaylimits_{x_A}^{x_B} F \dif x \\
	&= (-U(x_B)) - (-U(x_A)) \\
	&= U(x_A) - U(x_B)\\
	\therefore \dod{(-U(x))}{x} &= F \\
	\therefore F &= - \dod{U}{x} \\
	\therefore \overrightarrow{F} &= - \dod{U}{x} \hat{x} 
\end{align*}

\subsection{Potential Energy Corresponding to a General Force}

\begin{align*}
	\overrightarrow{F} &= (F_x (x, y, z), F_y (x, y, z), F_z (x, y, z)) \\
	\dif \overrightarrow{r} &= (\dif x, \dif y, \dif z)
\end{align*}

\begin{align*}
	\int\displaylimits_{(x_A, y_A, z_A)}^{(x_B, y_B, z_B)} \overrightarrow{F} \cdot \dif \overrightarrow{r} &= U_A - U_B \\
\end{align*}
\begin{align*}
	\therefore \int\displaylimits_{(x_A, y_A, z_A)}^{(x_B, y_A, z_A)} F_x \dif x &= (-U(x_A, y_A, z_A)) - (-U(x_B, y_A, z_A)) \\
	\therefore F_x &= \dpd{(-U)}{x} \\
	&= - \dpd{U}{x}
\end{align*}
\begin{align*}
	\therefore \int\displaylimits_{(x_A, y_A, z_A)}^{(x_B, y_B, z_A)} F_y \dif y &= (-U(x_A, y_A, z_A)) - (-U(x_B, y_B, z_A)) \\
	\therefore F_y &= \dpd{(-U)}{y} \\
	&= - \dpd{U}{y}
\end{align*}
\begin{align*}
	\therefore \int\displaylimits_{(x_A, y_A, z_A)}^{(x_B, y_B, z_B)} F_z \dif z &= (-U(x_A, y_A, z_A)) - (-U(x_B, y_B, z_B)) \\
	\therefore F_z &= \dpd{(-U)}{z} \\
	&= - \dpd{U}{z}
\end{align*}

\begin{definition}
	$\overrightarrow{F}$ is a conservative force iff 
	\begin{equation*}
		\exists U(x, y, z)$, s.t. $F_x = - \dpd{U}{x} , F_y = - \dpd{U}{y} , F_z = - \dpd{U}{z}
	\end{equation*}
	\begin{align*}
		\overrightarrow{F} &= \left(- \dpd{U}{x} , - \dpd{U}{y} , - \dpd{U}{z}\right) \\
		&= -\left(\dpd{}{x} , \dpd{}{y} , \dpd{}{z}\right) U \\
		&= - \overrightarrow{\nabla} U
	\end{align*}
\end{definition}

\begin{example}
	Show that $\overrightarrow{F}$ is conservative.
	\begin{equation*}
		\overrightarrow{F}(x, y) = (3 x^2 y^2 , 2 x^3 y)
	\end{equation*}
\end{example}

\begin{solution}
	\begin{align*}
		\dpd{U}{x} &= - 3 x^2 y^2 \\
		\therefore U &= \int\displaylimits -3 x^2 y^2 \dif x \\
		&= -3 \dfrac{x^3}{3} y^2 + c(y) \\
		&= x^3 y^2 + c(y) \\
		\dpd{U}{y} &= -2 x^3 y \\
		\therefore - x^3 (2y) + c'(y) &= -2 x^3 y \\
		\therefore c'(y) &= 0 \\
		\therefore c(y) &= \text{constant} \\
		\therefore U(x, y, z) &= -x^3 y^2 + c
	\end{align*}
\end{solution}

\begin{example}
	Show that $\overrightarrow{F}$ is conservative.
	\begin{equation*}
	\overrightarrow{F}(x, y) = (3 x^2 , 2 x^3 y)
	\end{equation*}
\end{example}

\begin{solution}
	\begin{align*}
		\dpd{U}{x} &= - 3 x^2 \\
		\therefore U &= \int\displaylimits -3 x^2 \dif x \\
		&= -3 \dfrac{x^3}{3} + c(y) \\
		&= x^3 + c(y) \\
		\dpd{U}{y} &= -2 x^3 y \\
		\therefore c'(y) &= -2 x^3 y 
	\end{align*}
	Therefore, $\overrightarrow{F}$ is non-conservative.
\end{solution}

\subsection{Line Integral Over a Closed Path}

If a force $\overrightarrow{F}$ is conservative, 
\begin{align*}
	\int\displaylimits_{\text{path 1}} \overrightarrow{F} \cdot \dif \overrightarrow{r} &= \int\displaylimits_{\text{path 2}} \overrightarrow{F} \cdot \dif \overrightarrow{r} \\
	\therefore 	\int\displaylimits_{\text{path 1}} \overrightarrow{F} \cdot \dif \overrightarrow{r} - \int\displaylimits_{\text{path 2}} \overrightarrow{F} \cdot \dif \overrightarrow{r} &= 0 \\
	\therefore 	\int\displaylimits_{\text{path 1}} \overrightarrow{F} \cdot \dif \overrightarrow{r} + \int\displaylimits_{\text{path 2}} \overrightarrow{F} \cdot ( - \dif \overrightarrow{r}) &= 0 \\
	\therefore \oint \overrightarrow{F} \cdot \dif \overrightarrow{r} &= 0
\end{align*}

\end{document}