\documentclass[fleqn, a4paper, 12pt]{article}
\usepackage{amsmath, amssymb, amsthm}
\usepackage{gensymb}
\usepackage{commath}
\usepackage{xcolor}
\usepackage{cancel}
\usepackage{siunitx}
\usepackage{tikz, pgfplots}
\usetikzlibrary{calc, hobby, patterns}
\usepackage{graphicx}
\usepackage{hyperref}
\usepackage{datetime}
\usepackage{ulem}
\usepackage{xfrac}
\usepackage{asymptote}
\usepackage{enumerate}
\setcounter{secnumdepth}{4}
\newcommand\numberthis{\addtocounter{equation}{1}\tag{\theequation}}

\newcommand{\AxisRotator}[1][rotate=0]{%
	\tikz [x=0.25cm,y=0.60cm,line width=.2ex,-stealth,#1] \draw (0,0) arc (-150:150:1 and 1);%
}

\theoremstyle{definition}
\newtheorem{example}{Example}
\newtheorem{definition}{Definition}

\theoremstyle{theorem}
\newtheorem{theorem}{Theorem}

\newenvironment{solution}
{\begin{proof}[Solution]\let\qed\relax}
	{\end{proof}}

\newcommand{\curl}{\mathrm{curl\,}}

%\renewcommand{\int_{min}^{max}}{\int\displaylimits_{min}^{max}}

%opening
\title{Lecture 15}
\author{Aakash Jog}
\date{\formatdate{18}{12}{2014}}

\begin{document}

\maketitle
%\setlength{\mathindent}{0pt}

\tableofcontents

\newpage
\section{Circular Motion Dynamics}

\begin{example}
	Find condition on $m_1$ and $m_2$ so that the loop will not jump up when the beads are released from rest from the top.\\
	\begin{tikzpicture}
		\coordinate (O) at (0,0);
		
		\def\r{2};
		\def\angle{60};
	
		\coordinate (bead1) at ({90-\angle}:\r);
		\coordinate (bead2) at ({90+\angle}:\r);
	
		\draw (O) circle [radius = \r];
	
		\draw [dashed] (O) -- ({90-\angle}:\r);
		\draw [dashed] (O) -- ({90+\angle}:\r);
		
		\fill (bead1) circle [radius = 2 pt];
		\fill (bead2) circle [radius = 2 pt];
		
		\draw (O)++(90:\r) -- ++(90:3);
		\draw [ultra thick] (O)++(90:{\r+3}) -- ++(0:5);
		\draw [ultra thick] (O)++(90:{\r+3}) -- ++(180:5);
	\end{tikzpicture}
\end{example}

\begin{solution}
	\hspace{1cm}\\
	\begin{tikzpicture}
		\coordinate (O) at (0,0);
		
		\def\r{4};
		\def\criticalangle{48};
		\def\angle{60};
		
		\coordinate (bead1) at ({90-\angle}:\r);
		\coordinate (bead2) at ({90+\angle}:\r);
		
		\draw (O) circle [radius = \r];
		
		\draw [dashed] (O) -- (90: \r);
		
		\draw [dashed] (O) -- ({90-\criticalangle}:\r);
			\draw (90:3) arc (90:{90-\criticalangle}:3) node [midway, fill = white] {$48.1 \degree$};
		\draw [dashed] (O) -- ({90+\criticalangle}:\r);
		
		\draw [dashed] (O) -- ({90-\angle}:\r);
			\draw (90:2) arc (90:{90-\angle}:2) node [midway, fill = white] {$\theta$};
		\draw [dashed] (O) -- ({90+\angle}:\r);
		
		\fill (bead1) circle [radius = 2 pt];
		\fill (bead2) circle [radius = 2 pt];
		
		\draw [-stealth, red](O) -- ++(-90:1) node [below] {$m_1 g$};
		
		\draw [-stealth, red](O)++(90:\r) -- ++(90:1) node [above] {$T$};
		
		\draw [-stealth, red] (bead1) -- ++({90-\angle}:2) node [above right] {$N$};
			\draw [-stealth, red] (bead1) -- ++(90:{2*(cos \angle)}) node [above] {$N \cos \theta$};
		\draw [-stealth, red] (bead2) -- ++({90+\angle}:2) node [above left] {$N$};
			\draw [-stealth, red] (bead2) -- ++(90:{2*(cos \angle)}) node [above] {$N \cos \theta$};
	\end{tikzpicture}
	The loop will jump up if
	\begin{align*}
		2 N \cos \theta > m_1 g
	\end{align*}
	Therefore, for the loop to not jump up,
	\begin{align*}
		2 N \cos \theta - m_1 g &> 0\\
		\therefore \dfrac{m_1}{m_2} &< 4 \cos \theta - 6 \cos^2 \theta
	\end{align*}
\end{solution}

\section{Rigid Body Mechanics}

\begin{example}
	A ladder is kept between 2 walls as shown.	Find the maximum angle for which the ladder does not slide.\\
	\begin{tikzpicture}
		%definition of origin
		\coordinate (O) at (0,0); 
		
		%definition of minimum and maximum values for axes
		\def\xMIN{0};
		\def\yMIN{0};
		\def\xMAX{6};
		\def\yMAX{6};
	
		%axes
		\draw [-stealth, lightgray] (\xMIN,0,0) -- (\xMAX,0) node [right] {$x$}; 
		\draw [-stealth, lightgray] (0,\yMIN,0) -- (0,\yMAX) node [above] {$y$};
	
		%coordinates
		\coordinate (ladder top) at (0,4);
		\coordinate (ladder bottom) at (3,0);
		\coordinate (COM) at ({3/2}, {4/2});
		
		%ladder
		\draw [ultra thick] (ladder top) -- (ladder bottom);
		
		\node at (ladder bottom) [xshift = -20, yshift= 10] {$\theta$};
		
		\node at (ladder top) [left] {$A$};
		\node at (ladder bottom) [below] {$C$};
		\node at (COM) [above right] {$B$};
		
		%forces
		\draw [-stealth, red] (ladder top) -- ++(0:1) node [right] {$N_2$};
		
		\draw [-stealth, red] (COM) -- ++(-90:1) node [below] {$mg$};
		
		\draw [-stealth, red] (ladder bottom) -- ++(90:1) node [above] {$N_1$};
		
		\draw [-stealth, red] (ladder bottom) -- ++(180:1) node [below] {$f_s$};
	\end{tikzpicture}
\end{example}

\begin{solution}
	\begin{align*}
		\sum \overrightarrow{F_{\text{ext}}} &= 0\\
		\intertext{Therefore,}
		N_1 &= mg\\
		N_2 &= f_s
	\end{align*}
	\begin{align*}
		\sum \overrightarrow{\tau_{\text{ext}, C}} &= 0\\
		\intertext{Therefore,}
		\dfrac{L}{2} \cos \theta m g - L \sin \theta N_2 &= 0
	\end{align*}
	Solving,
	\begin{align*}
		\cot \theta &\leq 2 \mu_s
	\end{align*}
\end{solution}

\begin{example}
	A box is kept on a surface and a force $F$ is applied to the top corner. Assuming the box does not slide, find the force required for the box to topple.
\end{example}

\begin{solution}
	\begin{tikzpicture}
		%definition of origin
		\coordinate (O) at (0,0); 
		
		%definition of minimum and maximum values for axes
		\def\xMIN{-2};
		\def\yMIN{-2};
		\def\xMAX{6};
		\def\yMAX{6};
		
		%ground
		\draw (\xMIN,0) -- (\xMAX,0); 
		
		%variables
		\def\b{3};
		\def\h{4};
		\def\x{1};
		
		%box
		\draw (0,0) rectangle (\b, \h);
		\node at (\b,0) [below] {$A$};
		
		%dimensions
		\draw [|<->|] (O)++(0,-2) -- ++(\b,0) node [midway, fill = white] {$b$};
		\draw [|<->|] (O)++(-1,0) -- ++(0,\h) node [midway, fill = white] {$h$};
		\draw [|<->|] (O)++({\b-\x},-1) -- ++(\x,0) node [midway, fill = white] {$x$};
		
		%forces
		\draw [-stealth, red] (\b, \h) -- ++(0:1) node [right] {$F$};
		\draw [-stealth, red] ({\b/2}, {\h/2}) -- ++(-90:1) node [below] {$mg$};
		\draw [-stealth, red] ({\b - \x}, 0) -- ++(90:1) node [above] {$N$};
		\draw [-stealth, red] ({\b - \x}, 0) -- ++(180:1) node [below left] {$f$};
	\end{tikzpicture}
	\begin{align*}
		\sum \overrightarrow{F_{\text{ext}}} &= 0\\
		\intertext{Therefore,}
		N &= mg\\
		F &= f
	\end{align*}
	\begin{align*}
		\sum \overrightarrow{\tau_{\text{ext}, A}} &= 0\\
		\intertext{Therefore,}
		-Nx + \dfrac{b}{2} m g - h F &= 0
	\end{align*}
	Solving,
	\begin{align*}
		F &= \dfrac{m g}{h} \left( \dfrac{b}{2} - x \right)
	\end{align*}
	The box will topple when $x = 0$, i.e. when
	\begin{align*}
		F &= \dfrac{m g b}{2h}
	\end{align*}
\end{solution}

\end{document}