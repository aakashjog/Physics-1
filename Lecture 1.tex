\documentclass[]{article}
\usepackage{amsmath, amssymb}
\usepackage{tikz}
\usepackage{datetime}
%opening
\title{Lecture 1}
\author{}
\date{\formatdate{28}{10}{2014}}

\begin{document}

\maketitle

\tableofcontents

\newpage
\section{Kinematics}

\begin{enumerate}
	\item	Galileo observed that in a v shaped arrangement of 2 inclined planes, a ball rolled down one rises to the same height, irrespective of the angles of the inclined planes. In the limiting case of this arrangement, where the angle of the second plane is 0, it can be deduced that force is not necessary for motion to continue.
\end{enumerate}



\newpage
\subsection{Examples}

\subsubsection{Example 1}

A car is standing at a traffic light. It starts accelerating at 2 m/s$^2$ for 6 seconds. It continues at a constant velocity for another 20 seconds. It approaches another traffic light, and decelerates at 3 m/s$^2$. \\

\noindent\resizebox{\textwidth}{!}
{
\begin{tikzpicture}

	\draw (0,0)  -- (6,12);
	\draw (6,12) -- (26,12);
	\draw (26,12) -- (30,0);

\end{tikzpicture}
}
\subsubsection{Example 2}

There is a train travelling at constant velocity of 108 km/hr. At $t=0$, the last car is detached. The car the decelerated with 4 m/s$^2$. The train continues with the same velocity. What is the distance between the car and the train, when the train stops. \\



\end{document}
