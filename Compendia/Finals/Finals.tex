\date{}
\documentclass[fleqn, a5paper]{amsart}
\usepackage[top = 2cm, bottom = 2cm, left = 2cm, right = 2cm, a5paper]{geometry}
\usepackage{amsmath, amssymb, amsthm}
\usepackage{marginnote}
\usepackage{gensymb}
\usepackage{commath}
\usepackage{xcolor}
\usepackage{cancel}
\usepackage{siunitx}
\usepackage{tikz, pgfplots}
	\usetikzlibrary{calc, hobby, patterns, intersections}
\usepackage{graphicx}
\usepackage{hyperref}
\usepackage{datetime}
\usepackage{ulem}
\usepackage{xfrac}
\usepackage{asymptote}
\usepackage{enumerate}
\usepackage{float}
\setcounter{secnumdepth}{4}
\newcommand\numberthis{\addtocounter{equation}{1}\tag{\theequation}}

\newcommand{\AxisRotator}[1][rotate=0]{%
	\tikz [x=0.25cm,y=0.60cm,line width=.2ex,-stealth,#1] \draw (0,0) arc (-150:150:1 and 1);%
}

\theoremstyle{definition}
\newtheorem{example}{Example}
\newtheorem{definition}{Definition}

\theoremstyle{theorem}
\newtheorem{theorem}{Theorem}

\newenvironment{solution}
{\begin{proof}[Solution]\let\qed\relax}
	{\end{proof}}

\newcommand{\curl}{\mathrm{curl\,}}

%\renewcommand{\int_{min}^{max}}{\int\displaylimits_{min}^{max}}

%opening
\title{Physics 1 : Compendium}
\author{Aakash Jog}

\begin{document}

\maketitle
%\setlength{\mathindent}{0pt}

\section{Non-Linear Motion}

\begin{align*}
	\overrightarrow{r} &= r \hat{r} \\
	\overrightarrow{v} = \dot{\overrightarrow{r}} &= \dot{r} \hat{r} + r \dot{\theta} \hat{\theta} \\
	\overrightarrow{a} = \ddot{\overrightarrow{r}} &= \left(\ddot{r} - r(\dot{\theta})^2\right) \hat{r} + \left(2 \dot{r} \dot{\theta} + r \ddot{\theta}\right) \hat{\theta}
\end{align*}

\section{Conservative Forces}
	\begin{align*}
		\curl \overrightarrow{F} &\doteq \overrightarrow{\nabla} \times \overrightarrow{F} \\
		&=
			\begin{vmatrix} 
				\hat{x} & \hat{y} & \hat{z} \\
				\dpd{}{x} & \dpd{}{y} & \dpd{}{z} \\
				F_x & F_y & F_z \\
			\end{vmatrix}
	\end{align*}
	If $\overrightarrow{F}$ is conservative, $\curl \overrightarrow{F} = 0$.

\section{Variable Mass Systems}

If a rocket is releasing gasses with velocity $u$ with respect to it,
\begin{equation*}
\dod{p}{t} = m \dod{v}{t} + \dod{m}{t} u
\end{equation*}

\section{Centres of Mass}

\begin{tabular}{l l}
	Solid hemisphere & $\left( \sfrac{3}{8} \right) R$\\
	Hollow hemisphere & $\left( \sfrac{1}{2} \right) R$\\
	Solid cone (from vertex) & $\left( \sfrac{3}{4} \right) h$\\
	Hollow cone (from vertex) & $\left( \sfrac{2}{3} \right) h$\\
\end{tabular}

\section{Moments of Inertia}

\begin{equation*}
	I = \int r^2 \dif m
\end{equation*}

\begin{tabular}{l l}
	Ring ($\perp$ to plane) & $m R^2$\\
	Disk ($\perp$ to plane) & $\left( \sfrac{1}{2} \right) m R^2$\\
	Solid sphere & $\left( \sfrac{2}{5} \right) m R^2$\\
	Hollow sphere & $\left( \sfrac{2}{3} \right) m R^2$\\
	Rod (centre) & $\left( \sfrac{1}{12} \right) m l^2$\\
	Rod (end) & $\left( \sfrac{1}{3} \right) m l^2$\\
	Cone (axis of symmetry) & $\left( \sfrac{3}{10} \right) m R^2$\\
\end{tabular}

\section{Accelerating Systems}

\begin{equation*}
	F_{\textnormal{centrifugal}} = - m \overrightarrow{\omega} \times \left( \overrightarrow{\omega} \times \overrightarrow{r} \right)
\end{equation*}
\begin{equation*}
	F_{\textnormal{coriolis}} = - 2 m \overrightarrow{\omega} \times \overrightarrow{v}
\end{equation*}

\section{Oscillations}

\subsection{Simple Oscillations}

\begin{equation*}
	\ddot{x} = - \omega^2 x
\end{equation*}

\begin{equation*}
	\omega_{\textnormal{physical pendulum}} = \sqrt{\dfrac{d_{\textnormal{axis,COM}} m g}{I_{\textnormal{axis}}}}
\end{equation*}

\subsection{Damped Oscillations}

\begin{equation*}
	\ddot{x} + \dfrac{\beta}{m} \dot{x} + {\omega_0}^2 x = 0
\end{equation*}

\begin{tabular}{c c}
	Strong damping & $\dfrac{\beta}{2m} > \omega_0$\\[2ex]
	Critical damping & $\dfrac{\beta}{2m} = \omega_0$\\[2ex]
	Weak damping & $\dfrac{\beta}{2m} < \omega_0$\\[2ex]
\end{tabular}\\
Oscillations occur in case of weak damping.\\

Let $\omega_1 = \sqrt{{\omega_0}^2 - \left( \dfrac{\beta}{2m} \right)^2}$.\\
For weak damping,
\begin{equation*}
	x = e^{-\sfrac{\beta}{2m} \cdot t} \left( \widetilde{A} \cos \omega_1 t + \widetilde{B} \sin \omega_1 t \right)
\end{equation*}
For critical damping,
\begin{equation*}
	x = e^{-\sfrac{\beta}{2m} \cdot t} \left( \widetilde{A} + \widetilde{B} t \right)
\end{equation*}
In case of strong damping,
\begin{equation*}
	x = \widetilde{A} e^{\left( -\sfrac{\beta}{2m} + \sqrt{-{\omega_1}^2} \right)} + \widetilde{B} e^{\left( -\sfrac{\beta}{2m} - \sqrt{-{\omega_1}^2} \right) t}
\end{equation*}

\subsection{Forced Oscillations}

\begin{align*}
	m \ddot{x} + \dfrac{k}{m} x &= \dfrac{F_0}{m} \cos \omega t\\
	\therefore \ddot{x} + \dfrac{k}{m} x &= \dfrac{F_0}{m} \cos \omega t
\end{align*}
Therefore, solving
\begin{align*}
	x &= A \cos \omega_0 t + B \sin \omega_0 t + \dfrac{F_0}{k - m \omega^2} \cos \omega t\\
	\therefore \dot{x} &= \omega_0 (-A \sin \omega_0 t + B \cos \omega_0 t) - \dfrac{F_0}{k - m \omega^2} \omega \sin \omega t
\end{align*}
Substituting initial conditions,
\begin{align*}
	x &= \dfrac{\dfrac{F_0}{m}}{\dfrac{k}{m} - \omega^2} (-\cos \omega_0 t + \cos \omega t)\\
	\intertext{Let $\dfrac{F_0}{m} = f_0$}
	\therefore x &= -\dfrac{2 f_0}{{\omega_0}^2 - \omega^2} \sin \left( \dfrac{\omega t - \omega_0 t}{2} \right) \sin \left( \dfrac{\omega t + \omega_0 t}{2} \right)\\
	\intertext{$\omega - \omega_0 = \Delta \omega$ and $\omega + \omega_0 \approx 2 \omega_0$}
	\therefore x &\approx \dfrac{2 f_0}{\Delta \omega \cdot 2 \omega_0} \sin \left( \dfrac{\Delta \omega}{2} t \right) \cdot \sin (\omega_0 t)
\end{align*}

\end{document}