\documentclass[fleqn]{article}
\usepackage{amsmath, amssymb, esdiff}
\usepackage{gensymb}
\usepackage{tikz, pgfplots}
\usepackage{datetime}
\usepackage{ulem}
\usepackage{enumerate}
\setcounter{secnumdepth}{4}
\newcommand\numberthis{\addtocounter{equation}{1}\tag{\theequation}}


%opening
\title{Lecture 2}
\author{}
\date{\formatdate{30}{10}{2014}}

\begin{document}
	
\maketitle
%\setlength{\mathindent}{0pt}

\tableofcontents

\newpage
\section{Newton's Laws of Motion}

\subsection{First Law}

Every body continues in its state of rest, or of uniform motion, unless there is a net force acting upon it. \label{Newton's First Law}\\

\subsection{Second Law}

\begin{equation}
	\sum \overrightarrow{F} = \diff{\overrightarrow{p}}{t} \label{Newton's Second Law}
\end{equation}
If the mass of a body is constant, 
\begin{equation}
	\sum \overrightarrow{F} = m \overrightarrow{a} \label{Newton's Second Law for constant mass}
\end{equation}

\subsection{Third Law}

The mutual actions of two bodies upon each other are always equal, and directed in opposite directions. \label{Newton's Third Law}\\

\end{document}
