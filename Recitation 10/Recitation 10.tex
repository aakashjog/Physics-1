\documentclass[fleqn, a4paper, 12pt]{article}
\usepackage{amsmath, amssymb, commath, amsthm}
\usepackage{gensymb}
\usepackage{datetime}
\usepackage{ulem}
\usepackage{enumerate}
\setcounter{secnumdepth}{4}

\newcommand\numberthis{\addtocounter{equation}{1}\tag{\theequation}}

\theoremstyle{definition}
\newtheorem{example}{Example}
\newtheorem{definition}{Definition}

\theoremstyle{theorem}
\newtheorem{theorem}{Theorem}

\newenvironment{solution}
{\begin{proof}[Solution]\let\qed\relax}
	{\end{proof}}
	
%opening
\title{Recitation 10}
\author{}
\date{\formatdate{31}{12}{2014}}

\begin{document}

\maketitle
%\setlength{\mathindent}{0pt}


\tableofcontents

\newpage
\section{}

\begin{tabular}{|c|c|}
	\hline
	Translational motion & Rotational Motion\\
	\hline
	$x$ & $\theta$\\
	$v$ & $\omega$\\
	$a$ & $\alpha$\\
	$\overrightarrow{p} = m \overrightarrow{v}$ & $\overrightarrow{L} = I \overrightarrow{\omega}$\\
	$\overrightarrow{F} = m \overrightarrow{a}$ & $\overrightarrow{\tau} = I \overrightarrow{\alpha}$\\
	$\overrightarrow{F} = \dod{\overrightarrow{p}}{t}$ & $\overrightarrow{\tau} = \dod{\overrightarrow{L}}{t}$\\
	\hline
\end{tabular}

\begin{example}
	A thin cylindrical chimney of length L is falling down, rotating around its base, point B, untill it breaks at some point P. Find the breaking point of the chimney.\\
\end{example}

\begin{solution}
	Let the angle between the chimney and the vetical be $\alpha$ and let the distance between $B$ and $P$ be $x$.\\
	The torque acting on the upper part of the chimney, about point $P$ is
	\begin{align*}
		\tau_p &= \left( \dfrac{m}{L} (L - x) \right) g \left( \dfrac{L - x}{2} \right) \sin \alpha\\
	\end{align*}
	Let
	\begin{align*}
		m_1 &= \dfrac{m}{L} (L - x)
	\end{align*}
	Therefore, the moment of inertia of the upper part of the chimney is
	\begin{align*}
		I_1 &= \dfrac{m_1(L - x)^2}{3}\\
		&= \dfrac{m}{L} \cdot \dfrac{(L - x)^3}{3}
	\end{align*}
	\begin{align*}
		\tau &= \tau_P + \tau(x)\\
		&= I_1 \ddot{\alpha}\\
		\therefore \tau(x) &= I_1 \ddot{\alpha}\\
		&= \left( \dfrac{m}{L} \right) \dfrac{(L - x)^3}{3} \ddot{\alpha} - \dfrac{M g}{2 L} (L - x)^2 \sin \alpha\\
		&= \left( \dfrac{m}{L} \right) \dfrac{(L - x)^3}{3} \left( \dfrac{3 g \sin \alpha}{2 L} \right) - \dfrac{M g}{2 L} (L - x)^2 \sin \alpha\\
		&= \left( \dfrac{M g \sin \alpha}{2 L} \right) \left( \dfrac{(L - x)^3}{L} - (L - x)^2 \right)
	\end{align*}
	Differentiating, $\tau(x)$ is maximum at $x = \dfrac{L}{3}$.
\end{solution}

\end{document}
