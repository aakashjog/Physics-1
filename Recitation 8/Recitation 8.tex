\documentclass[fleqn, a4paper, 12pt]{article}
\usepackage{amsmath, amssymb, commath, amsthm}
\usepackage{gensymb}
\usepackage{datetime}
\usepackage{ulem}
\usepackage{enumerate}
\setcounter{secnumdepth}{4}

\newcommand\numberthis{\addtocounter{equation}{1}\tag{\theequation}}

\theoremstyle{definition}
\newtheorem{example}{Example}
\newtheorem{definition}{Definition}

\theoremstyle{theorem}
\newtheorem{theorem}{Theorem}

\newenvironment{solution}
{\begin{proof}[Solution]\let\qed\relax}
	{\end{proof}}
	
%opening
\title{Recitation 8}
\author{}
\date{\formatdate{17}{12}{2014}}

\begin{document}

\maketitle
%\setlength{\mathindent}{0pt}

\tableofcontents

\newpage
\section{Centre of Mass}

\begin{example}
	Find the centre of mass of an eighth of a solid sphere.
\end{example}

\begin{solution}
	Consider an elemental mass $\dif m$ at $(r, \theta, \varphi)$.
	\begin{align*}
		x_{\text{COM}} &= \dfrac{\int\limits_{r = 0}^{R} \int\limits_{\theta = 0}^{\frac{\pi}{2}} \int\limits_{\varphi = 0}^{\frac{\pi}{2}} r \sin \theta \cos \varphi \dif V}{\iiint \dif V}\\
		&= \dfrac{\int\limits_{r = 0}^{R} \int\limits_{\theta = 0}^{\frac{\pi}{2}} \int\limits_{\varphi = 0}^{\frac{\pi}{2}} r \sin \theta \cos \varphi (r^2 \sin \theta \dif r \dif \theta \dif \varphi)}{\dfrac{1}{8} \cdot \dfrac{4}{3} \pi R^3}
	\end{align*}
	Therefore,
	\begin{equation*}
		\therefore x_{\text{COM}} = y_{\text{COM}} = z_{\text{COM}} = \dfrac{3}{8} R
	\end{equation*}
\end{solution}

\end{document}
