\documentclass[fleqn, a4paper, 12pt]{article}
\usepackage{amsmath, amssymb, commath, amsthm}
\usepackage{gensymb}
\usepackage{datetime}
\usepackage{ulem}
\usepackage{enumerate}
\setcounter{secnumdepth}{4}

\newcommand\numberthis{\addtocounter{equation}{1}\tag{\theequation}}

\renewcommand\thesubsubsection{\arabic{subsubsection}}

\theoremstyle{definition}
\newtheorem{example}{Example}
\newtheorem{definition}{Definition}

\theoremstyle{theorem}
\newtheorem{theorem}{Theorem}

\newenvironment{solution}
{\begin{proof}[Solution]\let\qed\relax}
	{\end{proof}}
	
%opening
\title{Recitation 12}
\author{}
\date{\formatdate{14}{1}{2015}}

\begin{document}

\maketitle
%\setlength{\mathindent}{0pt}

%\tableofcontents

%\newpage
\section*{Week 12 : Class Exercises}

\subsection*{Exercise 3}

\subsubsection{}

\begin{align*}
	m g &= k x_0\\
	\therefore x_0 &= \dfrac{m g}{k}
\end{align*}

\subsubsection{}

Let the distance about the equilibrium position be
\begin{align*}
	z &= x - x_0\\
	\therefore \ddot{z} &= \ddot{x}
\end{align*}
Therefore,
\begin{align*}
	\ddot{z} + \dfrac{k}{m} z &= 0\\
	\therefore z(t) &= A \cos \dfrac{k}{m} t + B \sin \dfrac{k}{m} t\\
	\intertext{According to the initial conditions,}
	z(t) &= -\dfrac{m v_0}{k} \sin \dfrac{k}{m} t
\end{align*}

\section*{Week 12 : Home Assignment}

\subsection*{Exercise 5}

\begin{align*}
	-kx - f &= m \ddot{x}
\end{align*}
Considering torques about the centre,
\begin{align*}
	R f &= \dfrac{m R^2}{2} \ddot{\theta}
\end{align*}
As the cylinder is purely rolling,
\begin{align*}
	\dot{x} &= \dot{\theta} R\\
	\ddot{x} &= \ddot{\theta} R
\end{align*}
Therefore,
\begin{align*}
	f &= \dfrac{m \ddot{x}}{2}\\
	\therefore -kx + \dfrac{m \ddot{x}}{2} &= m \ddot{x}\\
	\therefore \ddot{x} + \dfrac{2k}{3m} x &= 0\\
	\therefore \omega &= \sqrt{\dfrac{2k}{3m}}
\end{align*}
~\\
Also, as the cylinder is purely rolling, mechanical energy is conserved.\\
Therefore,
\begin{align*}
	E &= \dfrac{1}{2} k x^2 + \dfrac{1}{2} m \dot{x}^2 + \dfrac{1}{2} I {\dot{\theta}}^2\\
	&= \dfrac{1}{2} k x^2 + \dfrac{1}{2} m \dot{x}^2 + \dfrac{1}{4} m \dot{x}^2\\
	\therefore 0 &= k x \dot{x} + m \dot{x} \ddot{x} + \dfrac{1}{2} m \dot{x} \ddot{x}\\
	\therefore 0 &= \ddot{x} + \dfrac{2k}{3m} x\\
	\therefore \omega &= \sqrt{\dfrac{2k}{3m}}
\end{align*}

\section*{Week 13 : Class Exercises}

\subsection*{Exercise 1}

\begin{align*}
	v_{m, M} &= a \dot{\theta} \hat{\theta}\\
	&= a \dot{\theta} (\cos \theta \hat{i} + \sin \theta \hat{j})\\
	&= a \omega (\cos \theta \hat{i} + \sin \theta \hat{j})
\end{align*}
By COLM in the horizontal direction,
\begin{align*}
	0 &= m (v_m)_x + M (v_M)_x\\
	\therefore (v_m)_x &= - \dfrac{M}{m} (v_M)_x
\end{align*}
\begin{align*}
	(v_{m,M})_x &= (v_m)_x - (v_M)_x\\
	&= -\left( \dfrac{M}{m} + 1 \right) (v_M)_x\\
	\therefore - (v_M)_x \left( \dfrac{M + m}{m} \right) &= a \dot{\theta} \cos \theta \dot{i}\\
	\therefore (v_M)_x &= - a \dot{\theta} \cos \theta \dot{i} \left( \dfrac{m}{M + m} \right)\\
	\therefore (v_m)_x &= -\dfrac{M}{m} \cdot - a \dot{\theta} \cos \theta \dot{i} \left( \dfrac{m}{M + m} \right)\\
	&= \left( \dfrac{M}{M + m} \right) a \dot{\theta} \cos \theta \dot{i}
\end{align*}
\begin{align*}
	(v_m)_y &= a \dot{\theta} \sin \theta \hat{j}\\
\end{align*}
Therefore,
\begin{align*}
	v_m &= \left( \dfrac{M}{M + m} \right) a \dot{\theta} \cos \theta \dot{i} + a \dot{\theta} \sin \theta \hat{j}\\
	v_M &= - \left( \dfrac{m}{M + m} \right) a \dot{\theta} \cos \theta \hat{i}
\end{align*}

\end{document}